% Setting up the document class
\documentclass[12pt,a4paper]{article}

% Relevant packages to use
\usepackage{amssymb}
\usepackage{amsfonts}
\usepackage{amsmath}


% Begin the document
\begin{document}

%%%%%%%%%%%%%%%%%%%%%%%%%%%%%%%%%%%%%%%%
% First page
%%%%%%%%%%%%%%%%%%%%%%%%%%%%%%%%%%%%%%%%
\title{Propagators of warped-AdS fields}
\author{Artur Amorim}

\begin{abstract}
    These notes contain the derivation of the expression for the bulk-to-bulk and boundary-to-bulk propagators of scalar, vector and fermionic field in warped-AdS spaces.
\end{abstract}
\maketitle

%%%%%%%%%%%%%%%%%%%%%%%%%%%%%%%%%%%%%%%%
\section{Scalar Fields}
%%%%%%%%%%%%%%%%%%%%%%%%%%%%%%%%%%%%%%%%
Here we deduce expressions for the bulk-to-bulk and bulk-to-boundary propagators of scalar fields in warped-AdS spaces. We start with the simplest case of AdS and later we generalize for warped-AdS spaces.
\subsection{AdS space}
\subsection{Warped-AdS}
\subsubsection{Definition of the Bulk-to-Bulk propagator}
The action of the scalar field in a curved spacetime with a dilaton field turned on is
\begin{align}
    S_\phi = - \frac{1}{2} \int d^5 x \sqrt{-g} e^{-\Phi} \left( D_\mu \phi D^\mu \phi + m^2 \phi^2 \right)
\end{align}
where $\Phi$ is the dilaton, $\phi$ the scalar field and $D_\mu$ the covariant derivative. By using the action principle we find that the equations of motion of this field are
\begin{align}
    - D_\mu \left( e^{-\Phi} D^\mu \phi \right) + e^{-\Phi}m^2 \phi = 0
\end{align}
Assuming that the dilaton is only a function of $z$ one can show that
\begin{align}
    & D_\mu \Phi D^\mu \phi = \dot{\Phi} e^{-2A} \partial_z \phi \\ 
    & D_\mu D^\mu \phi = e^{-2A} \left( \partial^2_z + \partial^2  + 3 \dot{A} \partial_z \right) \phi
\end{align}
From this we get that the equation of motion of the scalar field is
\begin{align}
    \label{scalar_eom}
    e^{-\Phi - 2 A} \left[ \partial^2_z + \partial^2 + \left( 3\dot{A} - \dot{\Phi} \right) \partial_z - e^{2A} m^2 \right] \phi = 0
\end{align}
If we now add a term with a source $J$ to the action the field must satisfy the equation of motion
\begin{align}
    e^{-\Phi - 2 A} \left[ - \partial^2_z - \partial^2 - \left( 3\dot{A} - \dot{\Phi} \right) \partial_z + e^{2A} m^2 \right] \phi = J \left(x\right)
\end{align}
The general solution to this equation is
\begin{align}
    \phi \left( X \right) = \phi^0 \left(X\right) + \int d^5 \bar{X} \sqrt{-\bar{g}} G\left(X,\bar{X}\right) J \left(\bar{X}\right)
\end{align}
where $\phi^0$ is the solution of equation~\ref{scalar_eom} and G is the solution of the differential equation
\begin{align}
    e^{-\Phi - 2 A} \left[- \partial^2_z - \partial^2 - \left( 3\dot{A} - \dot{\Phi} \right) \partial_z + e^{2A} m^2 \right] G\left(X,\bar{X}\right)  = -i \frac{\delta^{(5)\left(X,\bar{X}\right)}}{\sqrt{-g}}
\end{align}
This equation can be written in Fourier space as
\begin{align}
   \left[- \partial_z \left( e^{3 A - \Phi} \partial_z \right) + m^2 e^{5 A - \Phi} + k^2 e^{3 A - \Phi} \right] g_k \left(z, \bar{z}\right) = - i \delta \left( z -\bar{z} \right)
\end{align}

%%%%%%%%%%%%%%%%%%%%%%%%%%%%%%%%%%%%%%%%
\subsubsection{Sturm-Liouville theory and equivalent Schrodinger problem}
%%%%%%%%%%%%%%%%%%%%%%%%%%%%%%%%%%%%%%%%

The last equation can be written as a Sturm-Liouville problem
\begin{align}
    \frac{d}{dz}\left( p\left(z\right) \frac{d y}{dz} \right) + q\left(z\right) y = - \lambda w\left(z\right) y
\end{align}
with
\begin{align}
    & p\left(z\right) =  e^{3 A - \Phi}, & q\left(z\right) =  -m^2  e^{5 A - \Phi} \\
    & w\left(z\right) = e^{3 A - \Phi}, & \lambda = - k^2
\end{align}
According with Sturm-Liouville theory there is a set of eigenvalues $\lambda_n$ with eigenfunctions $f_n$, that satisfy given boundary conditions, such that
\begin{align}
    \label{eq_eigenfunction}
    \frac{d}{dz}\left( p\left(z\right) \frac{d f_n}{dz} \right) + q\left(z\right) f_n = - \lambda_n w\left(z\right) f_n.
\end{align}
Moreover are a basis and orthonormal relation
\begin{align}
    \int d z f_n\left(z\right) f_m \left( z \right) w \left(z\right) = \delta_{n m}
\end{align}

In our case if we define
\begin{align}
    \label{sch_trans}
    f_n\left(z\right) = e^{\frac{\Phi - 3 A}{2}} \psi_n \left(z\right)
\end{align}
we can recast the last equation in Schrodinger form
\begin{align}
    & -\psi_n'' + V\left(z\right) \psi_n =  \lambda_n \psi_n, \\
    & V\left( z \right) = \frac{1}{4} \left( 9 \dot{A}^2 - 6 \dot{A} \dot{\Phi} + \dot{\Phi}^2 + 6 \ddot{A} - 2 \ddot{\Phi} + 4 m^2 e^{2A}\right)
\end{align}
We know from Quantum Mechanics that the eigenfunctions satisfy the completeness relation
\begin{align}
    \delta \left( z - \bar{z} \right) = \sum_n \psi_n \left(z\right) \psi_n \left(\bar{z}\right) 
\end{align}
Using~\ref{sch_trans} we can show
\begin{align}
    \label{delta_eigen_decomp}
    \delta \left( z - \bar{z} \right) = e^{3 A - \Phi} \sum_n f_n \left(z \right) f_n\left(\bar{z}\right)
\end{align}

%%%%%%%%%%%%%%%%%%%%%%%%%%%%%%%%%%%%%%%%
\subsubsection{Computation of $\tilde{G}\left(z,\bar{z},k\right)$}
%%%%%%%%%%%%%%%%%%%%%%%%%%%%%%%%%%%%%%%%

Because the eigenfunctions $f_n$ are a basis we write $\tilde{G}$ as a linear combination of these functions
\begin{align}
    g_k\left(z,\bar{z}\right) = \sum_n f_n\left(z\right) a_{k,n} \left(\bar{z}\right)
\end{align}
we get
\begin{align}
    \sum_n\left[- \partial_z \left( e^{3 A - \Phi} \partial_z \right) + m^2 e^{5 A - \Phi} + k^2 e^{3 A - \Phi} \right] f_n\left(z\right) a_{k,n} \left(\bar{z}\right) = - i \delta \left( z -\bar{z} \right).
 \end{align}
From~\ref{eq_eigenfunction} and~\ref{delta_eigen_decomp} and  we get
\begin{align}
    \sum_n \left(\lambda_n + k^2 \right) f_n\left(z\right) a_{k,n} \left(\bar{z}\right) = - i \sum_n f_n \left(z \right) f_n \left(\bar{z}\right)
 \end{align}
Using the orthogonality condition of the eigenfunctions $f_n$ we obtain
\begin{align}
    a_{k,n}\left(\bar{z}\right) = -i \frac{f_n\left(\bar{z}\right)}{k^2 + \lambda_n }
\end{align}
Note that there are poles when $k^2 = - \lambda_n$ for some n. We can identify these poles with a bound state of mass $m_n^2$. Given this the bulk-to-bulk propagator can be written as
\begin{align}
    G \left(z, x ; \bar{z}, \bar{x} \right) = - i \int \frac{d^4 k}{{\left(2 \pi \right)}^4} \sum_n \frac{f_n \left(z\right)f_n \left(\bar{z}\right)}{k^2 + m_n ^2 } e^{i k \cdot \left( x - \bar{x} \right)}
\end{align}
 %%%%%%%%%%%%%%%%%%%%%%%%%%%%%%%%%%%%%%%%
\section{Vector Fields}
%%%%%%%%%%%%%%%%%%%%%%%%%%%%%%%%%%%%%%%%

%%%%%%%%%%%%%%%%%%%%%%%%%%%%%%%%%%%%%%%%
\subsection{AdS space}
%%%%%%%%%%%%%%%%%%%%%%%%%%%%%%%%%%%%%%%%

%%%%%%%%%%%%%%%%%%%%%%%%%%%%%%%%%%%%%%%%
\subsection{Warped-AdS}
%%%%%%%%%%%%%%%%%%%%%%%%%%%%%%%%%%%%%%%%

%%%%%%%%%%%%%%%%%%%%%%%%%%%%%%%%%%%%%%%%
\section{Fermionic Fields}
%%%%%%%%%%%%%%%%%%%%%%%%%%%%%%%%%%%%%%%%

%%%%%%%%%%%%%%%%%%%%%%%%%%%%%%%%%%%%%%%%
\subsection{AdS space}
%%%%%%%%%%%%%%%%%%%%%%%%%%%%%%%%%%%%%%%%

%%%%%%%%%%%%%%%%%%%%%%%%%%%%%%%%%%%%%%%%
\subsection{Warped-AdS}
%%%%%%%%%%%%%%%%%%%%%%%%%%%%%%%%%%%%%%%%

\end{document}