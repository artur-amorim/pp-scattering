% Setting up the document class
\documentclass[12pt,a4paper]{article}

% Relevant packages to use
\usepackage{amssymb}
\usepackage{amsfonts}
\usepackage{amsmath, latexsym}
\usepackage{slashed}


% Begin the document
\begin{document}

%%%%%%%%%%%%%%%%%%%%%%%%%%%%%%%%%%%%%%%%
% First page
%%%%%%%%%%%%%%%%%%%%%%%%%%%%%%%%%%%%%%%%
\title{Propagators of warped-AdS fields}
\author{Artur Amorim}

\begin{abstract}
These notes contain the derivation of the expression for the bulk-to-bulk and boundary-to-bulk propagators of scalar, vector and fermionic field in warped-AdS spaces.
\end{abstract}
\maketitle

%%%%%%%%%%%%%%%%%%%%%%%%%%%%%%%%%%%%%%%%
\section{Scalar Fields}
%%%%%%%%%%%%%%%%%%%%%%%%%%%%%%%%%%%%%%%%
Here we deduce expressions for the bulk-to-bulk and bulk-to-boundary propagators of scalar fields in warped-AdS spaces. We start with the simplest case of AdS and later we generalize for warped-AdS spaces.
\subsection{AdS space}
\subsection{Warped-AdS}
\subsubsection{Definition of the Bulk-to-Bulk propagator}
The action of the scalar field in a curved spacetime with a dilaton field turned on is
\begin{align}
    S_\phi = - \frac{1}{2} \int d^5 x \sqrt{-g} e^{-\Phi} \left( D_\mu \phi D^\mu \phi + m^2 \phi^2 \right)
\end{align}
where $\Phi$ is the dilaton, $\phi$ the scalar field and $D_\mu$ the covariant derivative. By using the action principle we find that the equations of motion of this field are
\begin{align}
    - D_\mu \left( e^{-\Phi} D^\mu \phi \right) + e^{-\Phi}m^2 \phi = 0
\end{align}
Assuming that the dilaton is only a function of $z$ one can show that
\begin{align}
    & D_\mu \Phi D^\mu \phi = \dot{\Phi} e^{-2A} \partial_z \phi \\ 
    & D_\mu D^\mu \phi = e^{-2A} \left( \partial^2_z + \partial^2  + 3 \dot{A} \partial_z \right) \phi
\end{align}
From this we get that the equation of motion of the scalar field is
\begin{align}
    \label{scalar_eom}
    e^{-\Phi - 2 A} \left[ \partial^2_z + \partial^2 + \left( 3\dot{A} - \dot{\Phi} \right) \partial_z - e^{2A} m^2 \right] \phi = 0
\end{align}
If we now add a term with a source $J$ to the action the field must satisfy the equation of motion
\begin{align}
    e^{-\Phi - 2 A} \left[ - \partial^2_z - \partial^2 - \left( 3\dot{A} - \dot{\Phi} \right) \partial_z + e^{2A} m^2 \right] \phi = J \left(x\right)
\end{align}
The general solution to this equation is
\begin{align}
    \phi \left( X \right) = \phi^0 \left(X\right) + \int d^5 \bar{X} \sqrt{-\bar{g}} G\left(X,\bar{X}\right) J \left(\bar{X}\right)
\end{align}
where $\phi^0$ is the solution of equation~\ref{scalar_eom} and G is the solution of the differential equation
\begin{align}
    e^{-\Phi - 2 A} \left[- \partial^2_z - \partial^2 - \left( 3\dot{A} - \dot{\Phi} \right) \partial_z + e^{2A} m^2 \right] G\left(X,\bar{X}\right)  = -i \frac{\delta^{(5)\left(X,\bar{X}\right)}}{\sqrt{-g}}
\end{align}
This equation can be written in Fourier space as
\begin{align}
   \left[- \partial_z \left( e^{3 A - \Phi} \partial_z \right) + m^2 e^{5 A - \Phi} + k^2 e^{3 A - \Phi} \right] g_k \left(z, \bar{z}\right) = - i \delta \left( z -\bar{z} \right)
\end{align}

%%%%%%%%%%%%%%%%%%%%%%%%%%%%%%%%%%%%%%%%
\subsubsection{Sturm-Liouville theory and equivalent Schrodinger problem}
%%%%%%%%%%%%%%%%%%%%%%%%%%%%%%%%%%%%%%%%

The last equation can be written as a Sturm-Liouville problem
\begin{align}
    \frac{d}{dz}\left( p\left(z\right) \frac{d y}{dz} \right) + q\left(z\right) y = - \lambda w\left(z\right) y
\end{align}
with
\begin{align}
    & p\left(z\right) =  e^{3 A - \Phi}, & q\left(z\right) =  -m^2  e^{5 A - \Phi} \\
    & w\left(z\right) = e^{3 A - \Phi}, & \lambda = - k^2
\end{align}
According with Sturm-Liouville theory there is a set of eigenvalues $\lambda_n$ with eigenfunctions $f_n$, that satisfy given boundary conditions, such that
\begin{align}
    \label{eq_eigenfunction}
    \frac{d}{dz}\left( p\left(z\right) \frac{d f_n}{dz} \right) + q\left(z\right) f_n = - \lambda_n w\left(z\right) f_n.
\end{align}
Moreover they are a basis and orthonormal relation
\begin{align}
    \int d z f_n\left(z\right) f_m \left( z \right) w \left(z\right) = \delta_{n m}
\end{align}
The wavefunctions will be computed by imposing $f_n\left(0\right) = 0$ and $\partial_z f_n \left( z \rightarrow \infty \right) = 0$. The value of $\partial_z f_n \left( z \rightarrow \infty \right)$ at $z=0$ will be determined by the shooting method.

If we define
\begin{align}
    \label{sch_trans}
    f_n\left(z\right) = e^{\frac{\Phi - 3 A}{2}} \psi_n \left(z\right)
\end{align}
we can recast the last equation in Schrodinger form
\begin{align}
    & -\psi_n'' + V\left(z\right) \psi_n =  \lambda_n \psi_n, \\
    & V\left( z \right) = \frac{1}{4} \left( 9 \dot{A}^2 - 6 \dot{A} \dot{\Phi} + \dot{\Phi}^2 + 6 \ddot{A} - 2 \ddot{\Phi} + 4 m^2 e^{2A}\right)
\end{align}
We know from Quantum Mechanics that the eigenfunctions satisfy the completeness relation
\begin{align}
    \delta \left( z - \bar{z} \right) = \sum_n \psi_n \left(z\right) \psi_n \left(\bar{z}\right) 
\end{align}
Using~\ref{sch_trans} we can show
\begin{align}
    \label{delta_eigen_decomp}
    \delta \left( z - \bar{z} \right) = e^{3 A - \Phi} \sum_n f_n \left(z \right) f_n\left(\bar{z}\right)
\end{align}

%%%%%%%%%%%%%%%%%%%%%%%%%%%%%%%%%%%%%%%%
\subsubsection{Computation of $\tilde{G}\left(z,\bar{z},k\right)$}
%%%%%%%%%%%%%%%%%%%%%%%%%%%%%%%%%%%%%%%%

Because the eigenfunctions $f_n$ are a basis we write $\tilde{G}$ as a linear combination of these functions
\begin{align}
    g_k\left(z,\bar{z}\right) = \sum_n f_n\left(z\right) a_{k,n} \left(\bar{z}\right)
\end{align}
we get
\begin{align}
    \sum_n\left[- \partial_z \left( e^{3 A - \Phi} \partial_z \right) + m^2 e^{5 A - \Phi} + k^2 e^{3 A - \Phi} \right] f_n\left(z\right) a_{k,n} \left(\bar{z}\right) = - i \delta \left( z -\bar{z} \right).
 \end{align}
From~\ref{eq_eigenfunction} and~\ref{delta_eigen_decomp} and  we get
\begin{align}
    \sum_n \left(\lambda_n + k^2 \right) f_n\left(z\right) a_{k,n} \left(\bar{z}\right) = - i \sum_n f_n \left(z \right) f_n \left(\bar{z}\right)
 \end{align}
Using the orthogonality condition of the eigenfunctions $f_n$ we obtain
\begin{align}
    a_{k,n}\left(\bar{z}\right) = -i \frac{f_n\left(\bar{z}\right)}{k^2 + \lambda_n }
\end{align}
Note that there are poles when $k^2 = - \lambda_n$ for some n. We can identify these poles with a bound state of mass $m_n^2$. Given this the bulk-to-bulk propagator can be written as
\begin{align}
    G \left(z, x ; \bar{z}, \bar{x} \right) = - i \int \frac{d^4 k}{{\left(2 \pi \right)}^4} \sum_n \frac{f_n \left(z\right)f_n \left(\bar{z}\right)}{k^2 + m_n ^2 } e^{i k \cdot \left( x - \bar{x} \right)}
\end{align}
 %%%%%%%%%%%%%%%%%%%%%%%%%%%%%%%%%%%%%%%%
\section{Vector Fields}
%%%%%%%%%%%%%%%%%%%%%%%%%%%%%%%%%%%%%%%%

%%%%%%%%%%%%%%%%%%%%%%%%%%%%%%%%%%%%%%%%
\subsection{AdS space}
%%%%%%%%%%%%%%%%%%%%%%%%%%%%%%%%%%%%%%%%

%%%%%%%%%%%%%%%%%%%%%%%%%%%%%%%%%%%%%%%%
\subsection{Warped-AdS}
%%%%%%%%%%%%%%%%%%%%%%%%%%%%%%%%%%%%%%%%

%%%%%%%%%%%%%%%%%%%%%%%%%%%%%%%%%%%%%%%%
\section{Fermionic Fields}
%%%%%%%%%%%%%%%%%%%%%%%%%%%%%%%%%%%%%%%%

%%%%%%%%%%%%%%%%%%%%%%%%%%%%%%%%%%%%%%%%
\subsection{AdS space}
%%%%%%%%%%%%%%%%%%%%%%%%%%%%%%%%%%%%%%%%

%%%%%%%%%%%%%%%%%%%%%%%%%%%%%%%%%%%%%%%%
\subsubsection{Dirac equation in AdS}
%%%%%%%%%%%%%%%%%%%%%%%%%%%%%%%%%%%%%%%%
Here we solve the Dirac equation in a $d+1$-Euclidean AdS space. The Dirac equation in such a space is given by
\begin{align}
    \left( z \gamma^{\mu} \partial_\mu - \frac{d}{2} \gamma^5 - m_5 \right) \psi = 0
\label{dirac_eq}
\end{align}
where $\gamma^{\mu} = \{ \gamma^5, \gamma^{0}, \gamma^{1}, \gamma^{2}, \gamma^{3} \}$ and the chirality representation is assumed. These matrices satisfy $\gamma^{\mu} \gamma{\nu} + \gamma^{\nu} \gamma{\mu} = 2 \delta^{\mu \nu}$. If we apply $\gamma^{\mu} \partial_\mu$ to equation~\ref{dirac_eq} we get
\begin{align}
    \left[\partial^2_z + \Box - \frac{d}{z} \partial_z - \frac{1}{z^2} \left( m_5^2 - \frac{d^2}{4} - \frac{d}{2} - \gamma^{5} m_5 \right) \right] \psi \left(x \right) = 0
\label{dirac_eq_dec}
\end{align}
This equation can be split in two by considering eigenvectors of $\gamma^5 a^{\pm} = \pm a^{\pm}$. Their solutions, in Fourier space, are respectively
\begin{align}
    & z^{\frac{d+1}{2}} \left( C_1 K_{m_5 - \frac{1}{2}} \left( k z \right) + C_2 I_{m_5 - \frac{1}{2}} \left( k z \right)\right) \\
    & z^{\frac{d+1}{2}} \left( C_1 K_{m_5 + \frac{1}{2}} \left( k z \right) + C_2 I_{m_5 + \frac{1}{2}} \left( k z \right)\right) \notag
\end{align}
In the fermion Mathematica notebook we checked that the solution to the last equation that does not diverge when $z \rightarrow \infty$ is indeed
\begin{align}
    \psi\left(x\right) = \int \frac{d^d k}{{\left(2\pi\right)}^d} e^{- i\mathbf{k} \cdot \mathbf{x}} z^{\frac{d+1}{2}} \left( a^{+}\left( \mathbf{k} \right) K_{m_5-\frac{1}{2}} \left( k z \right) + a^{-}\left( \mathbf{k} \right) K_{m_5+\frac{1}{2}} \left( k z \right) \right) 
\end{align}
The spinors $a^{+}$ and $a^{-}$ are related by
\begin{align}
    a^{-} = \frac{i}{k} k^i \gamma^{i} a^{+}
\end{align}

%%%%%%%%%%%%%%%%%%%%%%%%%%%%%%%%%%%%%%%%
\subsubsection{Bulk-to-bulk propagator in AdS}
%%%%%%%%%%%%%%%%%%%%%%%%%%%%%%%%%%%%%%%%

Let us define 
\begin{align}
    &\phi^{\left( K \right)} \left(z, \mathbf{k} \right) = z^{\frac{d+1}{2}} \left[ K_{m+\frac{1}{2}} \left( k z \right) - \frac{i}{k} k^i \gamma^i K_{m-\frac{1}{2}} \left( k z \right)  \right] \\
    &\phi^{\left( I \right)} \left(z, \mathbf{k} \right) = z^{\frac{d+1}{2}} \left[ I_{m+\frac{1}{2}} \left( k z \right) + \frac{i}{k} k^i \gamma^i I_{m-\frac{1}{2}} \left( k z \right)  \right] \notag
\end{align}
The bulk-to-boundary propagator $ S \left(z, w \right) $needs to go to zero when $z, w \rightarrow 0$ and $z, w \rightarrow \infty $ and needs to satisfy the equation
\begin{align}
    \left(\slashed{D} - m_5\right) S\left(z,w\right) = \frac{1}{\sqrt{g}} \delta^{d+1} \left( z - w \right)
\label{spinor_propagotor_dirac}
\end{align}
We try the ansatz
\begin{align}
    S \left(z, w \right) = \int \frac{d^d \mathbf{k}}{{\left(2\pi\right)}^d} k e^{- i \mathbf{k}\cdot \left(\mathbf{z}-\mathbf{w}\right)} \left[ \theta \left( z - w \right)  \phi^{\left( K \right)} \left(z, \mathbf{k} \right) \mathcal{P}_{-} \bar{\phi}^{\left( I \right)} \left(w, -\mathbf{k} \right) - \right. \\  
    \left. - \theta\left( w - z \right) \phi^{\left( I \right)} \left(z, \mathbf{k} \right) \mathcal{P}_{-} \bar{\phi}^{\left( K \right)}\left(w, -\mathbf{k} \right) \right] \notag
\end{align}
In the fermion Mathematica notebook we proved this expression can be written as
\begin{align}
    S \left(z, w \right) = - z^{-\frac{1}{2}} \left[ \slashed{D} + \frac{1}{2}\gamma^5+ m_5 \right] \left[ G_{\frac{d}{2}+m_5 - \frac{1}{2}} \left(z,w\right) \mathcal{P}_{-} + G_{\frac{d}{2}+m_5 +\frac{1}{2}} \left(z,w\right) \mathcal{P}_{+}\right] w^{\frac{1}{2}}
\label{spinor_bulk_propagator}
\end{align}
where G is the scalar bulk-to-bulk propagator
\begin{align}
    G_{\Delta} \left(z, w \right) = \int \frac{d^d \mathbf{k}}{{\left(2\pi\right)}^d}  e^{- i \mathbf{k}\cdot \left(\mathbf{z}-\mathbf{w}\right)} \left[ \theta \left( z - w \right) K_{\nu} \left( k z \right) I_\nu \left( k w \right) - \right. \\  
    \left. - \theta\left( w - z \right) K_{\nu} \left( k w \right) I_\nu \left( k z \right) \right] \notag
\label{scalar_propagator}
\end{align}
Again in the mathematica notebook we showed that the expression~\ref{spinor_bulk_propagator} satifies the equation~\ref{spinor_propagotor_dirac} and hence it's the bulk-to-bulk propagator we were looking for.

%%%%%%%%%%%%%%%%%%%%%%%%%%%%%%%%%%%%%%%%
\subsubsection{boundary-to-bulk propagator in AdS}
%%%%%%%%%%%%%%%%%%%%%%%%%%%%%%%%%%%%%%%%
We now try to compute the boundary-to-bulk propagator of a bulk spinor field in AdS. It is related with the bulk-to-bulk propagator through
\begin{align}
    \Sigma_{\Delta_{+}} \left(z, \mathbf{x}\right) =  \lim_{\epsilon \to \ 0} - \epsilon^{-\Delta_{+} + \frac{1}{2}} S \left(z; \epsilon, \mathbf{x} \right)
\end{align}
From the expression of the scalar bulk propagator we can show that
\begin{align}
    & G_{m+\frac{d}{2} - \frac{1}{2}} \left(z, \epsilon \right) = \frac{\epsilon^{m_5 + \frac{d}{2} - \frac{1}{2}} z^{\frac{d}{2}} }{2^{m_5-\frac{1}{2}} \Gamma\left( m_5 + \frac{1}{2} \right)} \int \frac{d^d k}{{\left( 2 \pi \right)}^d} k^{m_5 - \frac{1}{2}} K_{m_5 - \frac{1}{2}}\left( k z \right) e^{- i \mathbf{k} \cdot \left( \mathbf{z} - \mathbf{w} \right)} \\
    & G_{m+\frac{d}{2} + \frac{1}{2}} \left(z, \epsilon \right) = \frac{\epsilon^{m_5 + \frac{d}{2} + \frac{1}{2}} z^{\frac{d}{2}} }{2^{m_5+\frac{1}{2}} \Gamma\left( m_5 + \frac{3}{2} \right)} \int \frac{d^d k}{{\left( 2 \pi \right)}^d} k^{m_5 + \frac{1}{2}} K_{m_5 + \frac{1}{2}}\left( k z \right) e^{- i \mathbf{k} \cdot \left( \mathbf{z} - \mathbf{w} \right)} \notag
\end{align}
With this result plus the expression of the spinor bulk propagator we can write the bulk-to-boundary of the spinor field as
\begin{align}
    & \Sigma_{\Delta_{+}} \left(z, \mathbf{x}\right) = z^{-\frac{1}{2}} \left[ \slashed{D} + \frac{1}{2}\gamma^5+ m_5 \right] \frac{z^{\frac{d}{2}}}{2^{m_5 - \frac{1}{2}}\Gamma\left( m_5 + \frac{1}{2} \right)} \times \\
    & \times  \int \frac{d^d k}{{\left( 2 \pi \right)}^d} k^{m_5 - \frac{1}{2}} K_{m_5 - \frac{1}{2}} \left( k z \right) e^{-i \mathbf{k} \cdot \left( \mathbf{z} - \mathbf{w} \right)} \mathcal{P}_{-} \notag
\end{align}
The integral in the last expression going to spherical coordinates and using the integral representation of the Bessel function
\begin{align}
   &\int d^d k = \int_0^{+ \infty} dk k^{d-1} \int_0^{\pi} d\theta {\left(\sin \theta\right)}^{d-2} \int d \Omega_{d-2} \\
   &\int_{0}^\pi d\theta {\left( \sin \theta \right)}^{d-2} \cos \left( k \left| \mathbf{z} - \mathbf{w} \right| \right) = \frac{\sqrt{\pi} \Gamma\left(\frac{d-1}{2}\right)2^{\frac{d}{2}-1}}{{\left(k \left| \mathbf{z} - \mathbf{w} \right| \right)}^{\frac{d}{2}-1}} J_{\frac{d}{2}-1} \left( k \left| \mathbf{z} - \mathbf{w} \right| \right)
\end{align}
we get an integral in k that was performed in the fermion notebook. Putting everything together we get
\begin{align}
    \Sigma_{\Delta_{+}} \left(z, \mathbf{x}\right) = \frac{\Gamma\left( m_5 + \frac{d-1}{2}\right)}{2 \pi^{\frac{d}{2}} \Gamma\left( m_5 + \frac{1}{2}\right)} z^{-\frac{1}{2}} \left[ \slashed{D} + \frac{1}{2}\gamma^5+ m_5 \right] {\left(\frac{z}{z^2+{\left| \mathbf{z} - \mathbf{w} \right|}^2}\right)}^{m_5 + \frac{d-1}{2}} \mathcal{P}_{-}
\end{align}
Since $\slashed{D} = z \gamma^{\mu} \partial_\mu - \frac{d}{2} \gamma5 $ after some algebra we get
\begin{align}
    \Sigma_{\Delta_{+}} \left(z, \mathbf{x}\right) = - U \left(z - w \right) K_\Delta \left(z, \mathbf{w}\right) \mathcal{P}_{-}
\end{align}
where $\Delta = m_5 + \frac{d}{2} + \frac{1}{2}$ 
\begin{align}
    & U\left(z - w \right) = \frac{\gamma^{\mu} \left( z_\mu - w_\mu \right)}{\sqrt{z}} \\
    & K_\Delta \left(z, \mathbf{w}\right) = \frac{\Gamma\left( \Delta \right)}{\pi^{\frac{d}{2}} \Gamma\left( \Delta - \frac{d}{2} \right)} {\left( \frac{z}{z^2 + {\left| \mathbf{z} - \mathbf{w} \right|}^2} \right)}^\Delta
\end{align}
%%%%%%%%%%%%%%%%%%%%%%%%%%%%%%%%%%%%%%%%
\subsection{Warped-AdS}
%%%%%%%%%%%%%%%%%%%%%%%%%%%%%%%%%%%%%%%%

\end{document}